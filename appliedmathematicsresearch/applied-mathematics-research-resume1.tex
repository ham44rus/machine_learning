\documentclass[aspectratio=169]{beamer}
\usepackage{amsmath, amssymb, bm}
\usepackage{mathtools}
\usepackage{graphicx}
\usepackage{xcolor}

% 日本語環境設定
\usepackage{CJKutf8}

\usetheme{Madrid}
\usecolortheme{default}
\setbeamertemplate{navigation symbols}{}

\begin{document}

\begin{CJK}{UTF8}{min}

\title{Wishart分布とその性質}
\subtitle{多変量正規分布の推定とBox-Cox変換}
\author{山北倫太郎}
\date{\today}

\begin{frame}
\titlepage
\end{frame}

\begin{frame}{目次}
\tableofcontents
\end{frame}

\section{Wishart分布}

\begin{frame}{7.1 はじめに}
\begin{itemize}
    \item $X=\begin{pmatrix}x_1' \\ \vdots \\ x_n'\end{pmatrix}$ は標本行列を表す 。
    \item $\bar{x}$ と $S$ は、それぞれ $\mu$ と $\Sigma$ の一貫性のある不偏推定量を提供する 。
    \begin{itemize}
        \item $n\bar{x}=X'1$ 
        \item $(n-1)S=X'X-n\bar{x}\bar{x}'$ 
    \end{itemize}
    \item 7.2節では、$x_1, \dots, x_n$ が独立同分布で $x \sim N_p(\mu, \Sigma)$ かつ $\Sigma>0$ の場合の $\mu$ と $\Sigma$ の最尤推定量が導出される 。
    \item $\bar{x}$ と $S$ の同時分布に関する基本的な結果が命題7.1で証明される 。
    \item 7.3節では、Wishart分布の基本的な特性が研究される 。
    \item 7.4節では、データの多変量正規性を高めるためのBox-Cox変換が提示される 。
\end{itemize}
\end{frame}

\begin{frame}{7.2 $\bar{x}$ と $S$ の同時分布}
\begin{itemize}
    \item 正規性がある場合、$\bar{x}$ と $S$ はいくつかの点で「最適」である 。
    \item $V=(n-1)S$ とする 。
    \item $X$ の確率密度関数は様々な方法で記述できる 。
    \begin{equation*}
    f(X) = (2\pi)^{-np/2} |\Sigma|^{-n/2} \exp\left[-\frac{1}{2} \sum_{i=1}^{n} (x_i - \mu)' \Sigma^{-1} (x_i - \mu)\right] \tag{7.1}
    \end{equation*}
    \begin{equation*}
    = (2\pi)^{-np/2} |\Sigma|^{-n/2} e^{-\frac{n}{2}\mu'\Sigma^{-1}\mu} \text{etr}\left[-\frac{1}{2}\text{tr }\Sigma^{-1}X'X + n\mu'\Sigma^{-1}\bar{x}\right] 
    \end{equation*}
    \begin{equation*}
    = (2\pi)^{-np/2} |\Sigma|^{-n/2} \text{etr}\left[-\frac{1}{2}[V + n(\bar{x} - \mu)(\bar{x} - \mu)']\Sigma^{-1}\right] 
    \end{equation*}
    \item $(X'X, \bar{x})$ (または $(S, \bar{x})$ のような一対一関数) は、$(\Sigma, \mu)$ の最小十分完全統計量である 。
    \item Rao-Blackwell/Lehmann-Scheff\'eの定理により、不偏推定量の中で $(\bar{x}, S)$ は最小分散を持つ 。
    \item $n-1 \ge p$ の場合の最尤推定量 (MLE) $\hat{\mu}$ と $\hat{\Sigma}$ を得るには、対数尤度関数を最小化する 。
    \begin{equation*}
    \ln |\Sigma| + \text{tr }\frac{1}{n}V\Sigma^{-1} + (\bar{x} - \mu)'\Sigma^{-1}(\bar{x} - \mu) \tag{7.2}
    \end{equation*}
    \item 最尤推定量は $\hat{\mu} = \bar{x}$ と $\hat{\Sigma} = \frac{1}{n}V$ である 。
\end{itemize}
\end{frame}

\begin{frame}{7.2 $\bar{x}$ と $S$ の同時分布 (続き)}
\begin{itemize}
    \item 一般的な結果として、$\bar{x}$ は $N_p(\mu, \Sigma/n)$ に従う 。
    \item 標本行列 $X$ を $Z$ を用いて表す。$X \overset{d}{=} ZA' + 1\mu'$, ここで $Z \sim N_n^p(0, I_n \otimes I_p)$ かつ $\Sigma=AA'$ 。
    \item $\bar{x}$ と $S_x$ の分布は $\bar{x}$ と $S_z$ の分布に等しい 。
    \item $P=n^{-1}11'$ と $Q=I-n^{-1}11'$ は直交射影である 。
    \item $PZ \perp QZ$ であるため、$\bar{z} \perp S_z$ である 。
    \item $Q=HH'$ は直交基底を与えるため、$(n-1)S_z = Z'HH'Z = U'U$ となる 。
\end{itemize}
\end{frame}

\begin{frame}{7.2 $\bar{x}$ と $S$ の同時分布 (続き)}
\begin{block}{定義 7.1 Wishart分布}
$W \sim W_p(m)$ ならば $W \overset{d}{=} \sum_{i=1}^{m} z_i z_i'$, ここで $z_i$ は独立同分布で $N_p(0, I)$ に従う。\\
$V \sim W_p(m, \Sigma)$ ならば $V \overset{d}{=} AWA'$, ここで $\Sigma=AA'$ かつ $W \sim W_p(m)$。
\end{block}

\begin{block}{命題 7.1}
$x_i$ が独立同分布で $N_p(\mu, \Sigma)$ に従う場合 ($i=1, \dots, n$)、
\begin{itemize}
    \item $\bar{x} \sim N_p(\mu, \Sigma/n)$ 
    \item $(n-1)S \sim W_p(n-1, \Sigma)$ 
    \item $\bar{x} \perp S$ 
\end{itemize}
\end{block}
\end{frame}

\begin{frame}{7.3 Wishart分布の性質 - 補題}
\begin{block}{補題 7.1}
$Z = (z_{ij}) \in \mathbb{R}^{n \times n}$ が独立同分布の $N(0, 1)$ に従う場合、$P(|Z| = 0) = 0$。
\end{block}

$n=1$ の場合は $z_{11}$ が絶対連続分布を持つため、結果は成立する。\\
$Z$ を以下のように分割する。
\[
Z = \begin{pmatrix} z_{11} & z_{12}' \\ z_{21} & Z_{22} \end{pmatrix}
\]
$Z_{22} \in \mathbb{R}^{(n-1) \times (n-1)}$ に対して結果が成立すると仮定すると、
\[
P(|Z|=0) = P(|Z|=0, |Z_{22}| \ne 0) + P(|Z|=0, |Z_{22}|=0)
\]
\[
= P(z_{11} = z_{12}'Z_{22}^{-1}z_{21}, |Z_{22}| \ne 0)
\]
\[
= E[P(z_{11} = z_{12}'Z_{22}^{-1}z_{21}, |Z_{22}| \ne 0 | z_{12}, z_{21}, Z_{22})] = 0
\]
\end{frame}

\begin{frame}{7.3 Wishart分布の性質 - 系}
\begin{block}{系 7.1}
$Z = (z_{ij}) \in \mathbb{R}^{n \times n}$ が独立同分布の $N(0, 1)$ に従う場合、$P(|Z| = t) = 0, \forall t$。
\end{block}

$P(|Z| = t) = E[P(z_{11} = z_{12}'Z_{22}^{-1}z_{21} + t/|Z_{22}|, |Z_{22}| \ne 0 | z_{12}, z_{21}, Z_{22})] = 0$。\\
補題7.1と系7.1は、$Z$ が任意の絶対連続分布を持つ場合にも有効である。
\end{frame}

\begin{frame}{7.3 Wishart分布の性質 - 命題}
\begin{block}{命題 7.3}
$W \sim W_p(m)$ かつ $m \ge p$ ならば、$W$ は確率1で非特異である。
\end{block}

$W \overset{d}{=} Z'Z$ であり、$Z' = (z_1, \dots, z_m)$ かつ $z_i$ は独立同分布の $N_p(0, I)$ に従う。\\
$\text{rank } W \overset{d}{=} \text{rank } Z'Z = \text{rank } Z \ge \text{rank } (z_1, \dots, z_p)$ は確率1で $p$ となる。\\
したがって、$\text{rank } W$ は確率1で $p$ となる。
\end{frame}

% ここに新しいスライドや内容を追加できます
\end{CJK}
\end{document}


